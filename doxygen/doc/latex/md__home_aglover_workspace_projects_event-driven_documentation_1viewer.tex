Visualise a stream of events from the cameras / from a pre-\/recorded sequence.

\subsection*{Description}

This application demonstrates how to visualise a stream of address events either from the cameras or from a pre-\/recorded sequence. The event stream is transmitted from the cameras (/zynq\+Grabber/v\+Bottle\+:o) to the v\+Pre\+Process (/v\+Pre\+Process/v\+Bottle\+:i), that removes salt-\/and-\/pepper noise from the event stream. The filtered stream (/v\+Pepper/v\+Bottle\+:o) is sent to v\+Framer (/v\+Framer/\+AE\+:i), that converts it to a yarpview-\/able image. The \char`\"{}images\char`\"{} from left (/v\+Framer/left) and right camera (/v\+Framer/right) are then sent to the yarp viewers (/view\+Ch0 and /view\+Ch1).

Here is a visualisation of the instantiated modules and connections.



\subsection*{Dependencies}

No special dependencies are required, all the required modules will be executed by the application.

\subsection*{How to run the application}

The application assumes you are connected to a {\itshape yarpserver} -\/ see \href{http://www.yarp.it/}{\tt http\+://www.\+yarp.\+it/} for basic instructions for using yarp.

Inside the {\itshape Application} folder in the yarpmanager gui, you should see an entry called {\itshape v\+View}. Double click and open it.

Now you are ready to run the application.

If you want to visualise events from the cameras, hit the {\itshape run} button and then {\itshape connect} on the yarpmanager gui.

To visualise events from a pre-\/recorded dataset, you can run {\itshape yarpdataplayer}.

Since {\itshape yarpdataplayer} opens the port with the same name as the real robot, make sure the same port is not running (or that you start an instance of the nameserver with your own namespace). 